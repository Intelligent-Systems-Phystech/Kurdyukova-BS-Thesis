\documentclass[12pt]{extarticle}

\usepackage[]{cite}
\usepackage{cmap}
\usepackage[T2A]{fontenc}
\usepackage[utf8]{inputenc}
\usepackage[english, russian]{babel}

% \usepackage{jmlda}
\newcommand{\hdir}{.}
\usepackage{amsmath, amsfonts,amssymb,mathrsfs}
\usepackage{graphicx}
\usepackage{minted}
\usepackage{hyperref}
\usepackage{mathtools}
\usepackage{tocloft}
\usepackage[linesnumbered,boxed]{algorithm2e}
% \usepackage{algorithm}
% \usepackage{algpseudocode}
% \usepackage[usenames]{color}
% \usepackage{colortbl}


\usepackage{graphicx, epsfig}
\usepackage{subfig}
\usepackage{color}

\usepackage{wrapfig}
\usepackage{float}
\usepackage{subfloat}
\usepackage{caption}
\usepackage{multirow}



\newtheorem{theorem}{Теорема}
\newtheorem{lemma}[theorem]{Лемма}
\newtheorem{definition}{Определение}
\newtheorem{remark}{Замечание}
\newenvironment{Proof} % имя окружения
    {\par\noindent{\bf Доказательство.}} % команды для \begin
    {\hfill$\scriptstyle\blacksquare$} % команды для \end

\DeclareMathOperator*{\argmax}{arg\,max}
\DeclareMathOperator*{\argmin}{arg\,min}
\newcommand{\Domain}{\mathcal{D}}
\newcommand{\supp}{\mathrm{supp}}
\newcommand{\diag}{\mathrm{diag}}
\newcommand{\bfw}{\mathbf{w}}
\newcommand{\bfv}{\mathbf{v}}
\newcommand{\bfx}{\mathbf{x}}
\newcommand{\bfz}{\mathbf{z}}
\newcommand{\bfX}{\mathbf{X}}
\newcommand{\bfy}{\mathbf{y}}
\newcommand{\bfb}{\mathbf{b}}
\newcommand{\bbr}{\mathbb{R}}
\newcommand{\bsigma}{\boldsymbol\Sigma}
\newcommand{\expectation}{\mathbb{E}}
\newcommand{\ceil}[1]{\lceil #1 \rceil}

\def\BibAuthor#1{\ruseng{\textit{#1}}}
\def\BibTitle#1{\ruseng{\textrm{#1}}}
\def\BibJournal#1{\ruseng{\textrm{#1}}{\textsl{#1}}}
\def\BibUrl#1{{\small\url{#1}}}
\def\BibHttp#1{{\small\url{http://#1}}}
\def\BibFtp#1{{\small\url{ftp://#1}}}
\def\BibDoi#1{doi:~{\small\url{http://dx.doi.org/#1}}}
\def\typeBibItem{\small\sloppy}


\textheight=22cm % высота текста
\textwidth=16cm % ширина текста
\oddsidemargin=0pt % отступ от левого края
\topmargin=-1.5cm % отступ от верхнего края
\parindent=24pt % абзацный отступ
\parskip=5pt % интервал между абзацами
\tolerance=2000 % терпимость к "жидким" строкам
\flushbottom % выравнивание высоты страниц

\begin{document}
\thispagestyle{empty}
\begin{center}
    \sc
        «Московский физико-технический институт \rm{(национальный исследовательский университет)}»\\
        Физтех-школа прикладной математики и информатики\\
        Кафедра <<Интеллектуальные системы>>
        %\\        при Вычислительном центре им. А. А. Дородницына РАН
        \\[35mm]
    \rm\large
        Курдюкова Антонина Дмитриевна\\[10mm]
    \bf\Large
		Снижение размерности фазового пространства в задачах канонического корреляционного анализа\\[10mm]
    \rm\normalsize
        03.03.01 -- Прикладные математика и физика\\[10mm]
    \sc
        Выпускная квалификационная работа бакалавра\\[10mm]
\end{center}
\hfill\parbox{85mm}{
    \begin{flushleft}
    \bf
        Научный руководитель:\\
    \rm
        д.ф.-м.н. Стрижов Вадим Викторович\\[3.9cm]
    \end{flushleft}
}
\begin{center}
    Москва\\
    2022
\end{center}

\newpage
\tableofcontents
\newpage

\begin{abstract}
Данная работа посвящена методам канонического корреляционного~анализа. Показано, что метод канонического корреляционного анализа является частным случаем метода сходящихся перекрестных отображений Сугихары. А вид прогностических моделей, соответствующих методу, представим в виде условия принадлежности двух аттракторов,  восстанавливаемых в исходном и целевом фазовых пространствах, к общей динамической системе. В работе рассмотрены метод PLS-CCA, метод Яушева-Исаченко с автоэнкодерами, NNPLS, seq2seq, Neural ODE. Сформулирован вариант теоремы о вложениях Такенса, пригодный для проверки того, что метод канонического корреляционного анализа или другой метод прогноза удовлетворяет условиям Сугихары. Решается прикладная задача в теоретической постановке. Рассматривается видеоряд ходьбы человека с акселерометром на руке.
\\
\bigskip
\noindent

\textbf{Ключевые слова}: \emph {снижение размерности, фазовое пространство, аттрактор, CCM, теорема Такенса о вложениях}
\end{abstract}
\newpage

%данные поля заполняются редакцией журнала
% \doi{10.21469/22233792}
% \receivedRus{01.01.2017}
% \receivedEng{January 01, 2017}

% \maketitle

% \cftchapterprecistoc




\section{Введение}



% \addcontentsline{toc}{section}{\protect\numberline{}Введение}




\subsection{Введение}
В работе исследуется связь между методапми корреляционного анализа и методом сходящихся перекрестных отображений (convergent cross mapping, CCM) \cite{sugihara1990nonlinear, sugihara2012detecting}.
Для ССМ нет способа выбора собственного подпространства, в котором аппроксимируется многообразие компакта и работает прогностическая модель. На текущий момент выбор собственного пространства осуществляется перебором по главным компонентам, например в \cite{usmanova}. Работа Исаченко [..] по PLS дает возможность перенести методы выбора подпространства с PLS на CCM.

\begin{definition}
Динамическая система -- множество элементов, для которого задана функциональная зависимость между временем и положением в фазовом пространстве каждого элемента системы
\end{definition}
Динамическая система представляет собой такую математическую модель некоего объекта, процесса или явления, в которой пренебрегают <<флуктуациями и всеми другими статистическими явлениям>>.

\begin{definition}
Фазовое пространство динамической системы -- совокупность всех допустимых состояний динамической системы.
\end{definition}

\begin{definition}
Траектория динамической системы в фазовом пространстве -- последовательность состояний
\end{definition}

\begin{definition}
Аттрактор -- компактное подмножество фазового пространства динамической системы, все траектории из некоторой окрестности которого стремятся к нему при времени, стремящемся к бесконечности. 
\end{definition}

\begin{definition}
Многообразие -- хаусдорфово топологическое пространство со счётной базой, каждая точка которого обладает окрестностью, гомеоморфной евклидову пространству $\bbr^n$
\end{definition}


\newpage

\subsection{Обзор литературы}




\section{Теоретическая часть}


\subsection{Метод сходящихся перекрестных отображений}
Метод сходящихся перекрестных отображений (convergent cross mapping, CMM) используется для исследования временных рядов на нанличие причинно--следственной связи. Корелляция не подразумевает причинно--следственную связь между рядами. Метод основан на теореме Такенса о вложениях. В общем случае многообразие аттрактора динамической системы может быть восстановлено по одной наблюдаемой~$\mathbf{X}$.

Согласно методу временной ряд $\mathbf{s}_1 = \{s_i^1 \}_{i=1}^{N_1}$ может быть восстановлен по ряду $\mathbf{s}_2 = \{s_i^2 \}_{i=1}^{N_2}$ только если временной ряд $\mathbf{s}_2$ связан с рядом $\mathbf{s}_1$. Временные ряды считаются связанными, если окрестность фазовой траектории $\mathbf{x}$ временного ряда $\mathbf{s}_1$ взаимно однозначно отображается в окрестность фазовой траектории $\mathbf{y}$ ряда $\mathbf{s}_2$. Иными словами, аттракторы $M_{\mathbf{X}}$ и $M_{\mathbf{Y}}$ наблюдаемых $\mathbf{X}$ и $\mathbf{Y}$ диффеоморфны, если $\mathbf{X}$ и $\mathbf{Y}$ принадлежат одной динамической системе.

\subsection{Метод проекций на латентные структуры}
Метод проекций на латентные структуры PLS~\cite{geladi1988notes, hoskuldsson1988pls} используют для нахождения фундаментальных зависимостей между двумя матрицами $\mathbf{X}$ и $\mathbf{Y}$. Отбираются наиболее значимые прихнаки. Новые признаки являются их линейными комбинациями. Осуществляется переход в фазовое пространство меньшей размерности. Метод PLS позволяет найти фазовое подпространство, в котором наблюдается связь между главными компонентами исходных временных рядов. Это позволяет исследовать наличие связи между временными рядами. 

Пусть $\mathbf{X}\in\bbr^{m\times n}$ и $\mathbf{Y}\in\bbr^{m\times r}$ ~--- матрицы двух фазовых пространств, построенных по временному ряду $\mathbf{s}_1$ и $\mathbf{s}_2$ соответственно. Требуется построить прогноз временного ряда $\mathbf{s}_2$ с учетом связи с временным рядом$\mathbf{s}_1$. Предполагается линейная зависимость между строками $\mathbf{X}$ и $\mathbf{Y}$:
\begin{equation}
    \mathbf{Y}_i = \mathbf{X}_i\cdot\mathbf{\Theta} + \mathbf{\varepsilon} \quad \mathbf{Y}_i\in\bbr^r,\;\mathbf{Y}_i\in\bbr^n,\; i = 1,\ldots,m,
    \label{eq:linear}
\end{equation}
где $\mathbf{\Theta}$ ~--- матрица весов линейной зависимости,\; $\mathbf{\varepsilon}$ ~--- вектор ошибок.

Ошибка вычисляется по формуле:
\begin{equation}
    S(\mathbf{\Theta}, \mathbf{X}, \mathbf{Y}) = \|\mathbf{Y} - \mathbf{X}\cdot\mathbf{\Theta}\|_2^2
    \label{eq:error}
\end{equation}

Алгоритм PLS находит матрицы $\mathbf{T}, \mathbf{P}, \mathbf{Q}$, с помощью которых осуществляется переход в латентное пространство согласно формулам:
\begin{equation}
    \mathbf{X} = \mathbf{T}\cdot \mathbf{P}^{\mathsf{T}} + \mathbf{F}
    \label{eq:latent_X}
\end{equation}
\begin{equation}
    \mathbf{Y} = \mathbf{T}\cdot \mathbf{Q}^{\mathsf{T}} + \mathbf{E}
    \label{eq:latent_Y}
\end{equation}
Матрица $\mathbf{T}$ наилучшим образом описывает $\mathbf{X}$ и $\mathbf{Y}$. Ее столбцы ортогональны. Матрицами $\mathbf{P}$ и $\mathbf{Q}$ определяется переход из латентного пространства в исходное. Матрицы $\mathbf{X}$ и $\mathbf{Y}$ ~--- матрицы невязок.

Алгоритм PlS также позволяет определить матрицу $\mathbf{W}$, с помощью которой рассчитывается матрица весов $\mathbf{\Theta}$:
\begin{equation}
    \mathbf{\Theta} = \mathbf{W}(\mathbf{P}^{\mathsf{T}}\mathbf{W})^{-1}\mathbf{Q}^{\mathsf{T}}
    \label{eq:Q}
\end{equation}

\subsection{Постановка задачи}


\begin{theorem}
\label{th:Aduenko_necessary} 

\end{theorem}

\begin{lemma}
\label{lem:Aduenko_necessary} 

\end{lemma}

\begin{Proof}%[Доказательство]



\end{Proof}

\newpage
\section{Результаты экспериментов}



\newpage
\subsection{Вычислительный эксперимент}





\newpage
\section{Заключение}



\bibliographystyle{unsrt}
\bibliography{biblio}


\end{document}
